%== .basic ==%
\begin{itemize}
\item Edgar Allen Poe
\item Sheri S. Tepper
\item Bill Bryson
\end{itemize}

%== .with_title ==%
\begin{itemize}
\item Edgar Allen Poe
\item Sheri S. Tepper
\item Bill Bryson
\end{itemize}

%== .with_id_and_role ==%
\begin{itemize}
\item Edgar Allen Poe
\item Sheri S. Tepper
\item Bill Bryson
\end{itemize}

%== .max_nesting ==%
\begin{itemize}
\item level 1
\begin{itemize}
\item level 2
\begin{itemize}
\item level 3
\begin{itemize}
\item level 4
% LaTeX does not support more than four levels
%\begin{itemize}
%\item level 5
%\end{itemize}
\end{itemize}
\end{itemize}
\item level 2
\end{itemize}
\end{itemize}

%== .complex_content ==%
\begin{itemize}
\item Every list item has at least one paragraph of content,
which may be wrapped, even using a hanging indent.
Additional paragraphs or blocks are adjoined by putting
a list continuation on a line adjacent to both blocks.
\begin{description}
\item[list continuation]a plus sign ({\tt +}) on a line by itself
\end{description}
\item A literal paragraph does not require a list continuation.
\begin{verbatim}
$ gem install asciidoctor
\end{verbatim}
\end{itemize}

%== .checklist ==%
\begin{itemize}
\item checked
\item also checked
\item not checked
\item normal list item
\end{itemize}
